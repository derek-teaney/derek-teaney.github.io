\newif\ifanswers
%\answerstrue
\documentclass[12pt]{article}

\RequirePackage[top=1in,bottom=1in,left=1in,right=1in]{geometry}
\usepackage{graphicx,graphics}
\usepackage{bm,mathrsfs, amssymb,amsmath,accents}
%\usepackage[opta]{optional}
\usepackage[colorlinks=true]{hyperref}

\def\half{{\textstyle\frac{1}{2}}}
\def\third{{\textstyle\frac{1}{3}}}
\def\threehalf{{\textstyle\frac{3}{2}}}
\def\twothird{{\textstyle\frac{2}{3}}}
\def\quarter{{\textstyle\frac{1}{4}}}
\def\fifth{{\textstyle\frac{1}{5}}}
\def\sixth{{\textstyle\frac{1}{6}}}
\def\mring#1{\accentset{\circ}{#1}}

\def\ul#1{\underline{#1}}
\def\options#1#2{{\opt{opta}{#1}\opt{optb}{#2}}}
\def\F{{\bm F}}
\def\E{{\bm E}}
\def\da{{\rm d}{\bf a}}
\def\B{{\bm B}}
\def\A{{\bm A}}
\def\n{{\bm n}}
\def\pot{{\varphi}}
\def\x{{\bm x}}
\def\xh{{\hat{\bf x}}}
\def\yh{{\hat{\bf y}}}
\def\zh{{\hat{\bf z}}}
\def\rh{{\hat{\bm r}}}
\def\p{{\bm p}}
\def\r{{\bm r}}
\def\j{{\bf j}}
\def\k{{\bm k}}
\def\rhat{{\bm \hat{r} }}
\def\S{{\bm S}}
\def\dl{{\rm d}{\bm \ell}}
\def\dV{{\rm d}^3r}
\def\n{{\bm n}}
\def\v{{\bm v}}
\def\phmu{\phantom{\mu}}

\def\ppx#1{\frac{\partial}{\partial #1}}

\def\st{\begin{equation}}
\def\stp{\end{equation}}
\def\bg{\begin{eqnarray}}
\def\nd{\end{eqnarray}}
\def\Eq#1{Eq.~(\ref{#1})}
\def\llangle{\left\langle }
\def\rrangle{\right\rangle }

\newcommand*{\handfig}[1]{\vspace{1.5in}\begin{center}Figure: {#1}\end{center}} 

\newcommand*{\anyproblem}[1]{\newpage\subsection*{#1}}
\newcommand*{\problem}[1]{\stepcounter{ProblemNum} %
   \anyproblem{Problem \theProblemNum. \; #1}}

\newcommand*{\soln}[1]{\subsubsection*{#1}}
\newcommand*{\solution}{\soln{Solution:}}
\renewcommand*{\part}{\stepcounter{SubProblemNum} %
  \soln{Part (\theSubProblemNum)}}

\newcounter{ProblemNum}
\newcounter{SubProblemNum}[ProblemNum]
 
\renewcommand{\theProblemNum}{\arabic{ProblemNum}}
\renewcommand{\theSubProblemNum}{\alph{SubProblemNum}}
  
\renewcommand{\theenumi}{(\alph{enumi})}
\renewcommand{\labelenumi}{\theenumi}
\renewcommand{\theenumii}{\roman{enumii}}


%\newenvironment{solution}%
%{\noindent\ignorespaces {\bf Solution:}\\}%
%{\par\noindent%
%\ignorespacesafterend}


\begin{document}

\problem{One liners}
\begin{enumerate}
\item Starting from the Maxwell equations for $F^{\mu\nu}$ and
the definition of $F^{\mu\nu}$,  derive 
the wave equation $-\Box A^{\mu} = J^{\mu}/c$.
\item Starting from the maxwell equations for $F^{\mu\nu}$ in covariant form,
show that we must have $\partial_{\mu} J^{\mu} = 0$ for consistency.
\item (This is two lines) Show that the energy conserivation and force laws
\begin{align}
    \frac{dE_\p}{dt} =& q\E\cdot \v_p \\
    \frac{d\p}{dt} =& q (\E + \frac{{\bm v}_\p}{c} \times \B)
\end{align}
can be written covariantly
\st
\label{foo}
 \frac{dP^{\mu}}{d\tau} = F^{\mu\nu} u_{\nu}/c
\stp
Note that $E_\p$ (the energy of the particle) is different from $\E$ the
electric field.
\item From \Eq{foo} show that $P_{\mu}P^{\mu}$ is constant in time.

\item Show that $F_{\mu\nu} = \partial_\mu A_{\nu} - \partial_\nu A_{\mu}$  
    is invariant under the gauge transform
    \st
                A_{\mu}  \rightarrow A_{\mu} + \partial_{\mu} \Lambda(X)
    \stp
    where $\Lambda$ is an arbitrary function of $X = (t,\r)$.

\item Given $F^{\mu\nu}$ the only two Lorentz invariant quantities 
    are $F_{\mu\nu} F^{\mu\nu}$ and $F_{\mu\nu} \tilde F^{\mu\nu}$. 
Evaluate these two invariants in terms of $\E$ and $\B$\footnote{{\tiny answers: $2(B^2 - E^2)$ and $-4E\cdot B$}}
\end{enumerate}

\end{document}


