
\newif\ifanswers
%\answerstrue
\documentclass[12pt]{article}

\RequirePackage[top=1in,bottom=1in,left=1in,right=1in]{geometry}
\usepackage{graphicx,graphics}
\usepackage{bm,mathrsfs, amssymb,amsmath,accents}
%\usepackage[opta]{optional}
\usepackage[colorlinks=true]{hyperref}

\def\half{{\textstyle\frac{1}{2}}}
\def\third{{\textstyle\frac{1}{3}}}
\def\threehalf{{\textstyle\frac{3}{2}}}
\def\twothird{{\textstyle\frac{2}{3}}}
\def\quarter{{\textstyle\frac{1}{4}}}
\def\fifth{{\textstyle\frac{1}{5}}}
\def\sixth{{\textstyle\frac{1}{6}}}
\def\mring#1{\accentset{\circ}{#1}}

\def\ul#1{\underline{#1}}
\def\options#1#2{{\opt{opta}{#1}\opt{optb}{#2}}}
\def\F{{\bm F}}
\def\E{{\bm E}}
\def\da{{\rm d}{\bf a}}
\def\B{{\bm B}}
\def\A{{\bm A}}
\def\n{{\bm n}}
\def\pot{{\varphi}}
\def\x{{\bm x}}
\def\xh{{\hat{\bf x}}}
\def\yh{{\hat{\bf y}}}
\def\zh{{\hat{\bf z}}}
\def\rh{{\hat{\bm r}}}
\def\p{{\bm p}}
\def\r{{\bm r}}
\def\j{{\bm j}}
\def\k{{\bm k}}
\def\rhat{{\bm \hat{r} }}
\def\S{{\bm S}}
\def\dl{{\rm d}{\bm \ell}}
\def\dV{{\rm d}^3r}
\def\n{{\bm n}}
\def\v{{\bm v}}

\def\ppx#1{\frac{\partial}{\partial #1}}

\def\st{\begin{equation}}
\def\stp{\end{equation}}
\def\bg{\begin{eqnarray}}
\def\nd{\end{eqnarray}}
\def\Eq#1{Eq.~(\ref{#1})}
\def\llangle{\left\langle }
\def\rrangle{\right\rangle }

\newcommand*{\handfig}[1]{\vspace{1.5in}\begin{center}Figure: {#1}\end{center}} 

\newcommand*{\anyproblem}[1]{\newpage\subsection*{#1}}
\newcommand*{\problem}[1]{\stepcounter{ProblemNum} %
   \anyproblem{Problem \theProblemNum. \; #1}}

\newcommand*{\soln}[1]{\subsubsection*{#1}}
\newcommand*{\solution}{\soln{Solution:}}
\renewcommand*{\part}{\stepcounter{SubProblemNum} %
  \soln{Part (\theSubProblemNum)}}

\newcounter{ProblemNum}
\newcounter{SubProblemNum}[ProblemNum]
 
\renewcommand{\theProblemNum}{\arabic{ProblemNum}}
\renewcommand{\theSubProblemNum}{\alph{SubProblemNum}}
  
\renewcommand{\theenumi}{(\alph{enumi})}
\renewcommand{\labelenumi}{\theenumi}
\renewcommand{\theenumii}{\roman{enumii}}


%\newenvironment{solution}%
%{\noindent\ignorespaces {\bf Solution:}\\}%
%{\par\noindent%
%\ignorespacesafterend}


\begin{document}


%\problem{Dipole two ways}
%Consider two charges $\pm q$ moving a short distance on $\ell$ the $z$ axis
%forming  a dipole, i.e.
%\begin{align}
%  z_{+}(t) =& \ell/2 \cos(\omega t) \\
%  z_{-}(t) =& -\ell/2 \cos(\omega t)
%\end{align}
%This charged has dipole moment $p = q \ell \cos(\omega t)$  directed in the $z$ direction
%\begin{enumerate}
%\item Write down the electric field as a function of time.
%\item
%   By treating the time
%dependent electric field as a ``displacement current"
%\st
%  \frac{{\bm j}_D }{c}= \frac{1}{c} \partial_t \E,
%\stp 
%      determine the magnetic field 
%using Ampere's Law 
%\st
%  \oint \B\cdot d{\bm \ell} = \int_S d{\bm a} \cdot \frac{{\bm j}_D }{c} 
%\stp
%with an appropriate Amperian loop.
%      Assume that ${\bm B}=B(r,\theta) \, \hat{\phi}$ which 
%      can be justified from the symmetry of the problem. You should find
%      \st
%           {\B} =  \frac{\dot \p \times \n}{4\pi r^2 c}
%      \stp

%\item  Consider the Biot Savat Law
%   \st
%       \B(\r) = \int d^3\r_0 \frac{{\bm j(\r_0)}}{c} \times \frac{(\r - \r_0) }{4\pi |\r - \r_0|^3 }
%    \stp
%      Show that if ${\bm j}$ is curl free $\nabla \times {\bm j} =0$ 
%      the magnetic field is zero.  Explain why the displacement current from an electrostatic field
%      does not need to be included in the Biot-Savat Law. 

%      Thus we see that the \emph{electrostatic} displacement current
%      is necessary for consistency of Ampere's Law but does not actually
%      produce a magnetic field.


%\item 
%   Show (using a high-school physics argument) that the current integrated over small volume $\Delta V$
%      surrounding the charges is
%\st
% {\bm j} \Delta V = \partial_t \p
%\stp
%Use the Biot-Savat Law 
%to determine the magnetic
%field.   It should agree with $(b)$. 


%%\item Recall the a magnetic dipole has a vector 
%%potential
%%\st
%%   {\bm A} = \frac{{\bm m} \times \hat {\bm r} }{4\pi r^3}
%%\stp
%%and $\nabla \times {\bm A} = {\bm B}$. Using a formal analogy 
%%to this familiar result, determine the magnetic field from a time dependent 
%%electric dipole to first order in $1/c$.

%%\item For a constant vector   ${\v}$ times a scalar \phi(\r) show 
%%   that
%%   \st
%%      \nabla \times (\v \phi) =  \v \times (-\nabla \phi) \, .
%%   \stp
%%   Use this to  show that the vector potential in a particular gauge
%%   takes the form
%%      \st
%%             \frac{\partial_t {\bm p} }{4\pi r}
%%      \stp
%%      Show that this is the Lorentz gauge and construct a gague transformation
%%      to find the vector potential in the Coulomb gauge

%\item At what radius do the  electric fields  part (a) 
%   and the magnetic fields of part (b), (d) become
%   equal in magnitude. At this radius the approximation scheme is no
%   longer valid.

%%\item For a constant vector   ${\v}$ times a scalar \phi(\r) show 
%%   that
%%   \st
%%      \nabla \times (\v \phi) =  \v \times (-\nabla \phi) \, .
%%   \stp
%%   Use this to  show that the vector potential in a particular gauge
%%   takes the form
%%      \st
%%             \frac{\partial_t {\bm p} }{4\pi r}
%%      \stp
%%      Show that this is the Lorentz gauge. 

%%\item Construct a gague transformation 
%%      to find the vector potential in the Coulomb gauge.
%%      \st
%%          \A = \frac{ 
%%      \stp
%%      Hint try a gauge transformation

%\end{enumerate}
%\newpage

\problem{Dipole from potentials to order $1/c^2$}
      This continues the ``Dipole two ways'' problem  from the homework

\begin{enumerate}
%\item For a constant vector   ${\v}$ times a scalar $f(\r)$ show 
%   that
%   \st
%      \nabla \times (\v f(\r)) =  \v \times (-\nabla f(\r)) \, .
%   \stp
%   Use this to  guess that the vector potential in a particular gauge
%   takes the form
%      \st
%            \label{Aguess}
%            {\bm A}  = \frac{\partial_t {\bm p} }{4\pi r c}
%      \stp
%      As we will see this is the  Lorenz gauge

%      Show that this is the Lorenz gauge by 
%      evaluating
%      \st
%         \frac{1}{c} \partial_t \pot + \nabla \cdot \A = 0 
%      \stp


   \item (Optional) Starting from the Maxwell equations derive/write down  the equations for ($\phi,\A$)
   in the Lorenz gauge 
      \st
          \frac{1}{c} \partial_t \varphi + \nabla \cdot \A = 0 
      \stp
      and Coulomb gauges
      \st
          \nabla \cdot \A = 0 
      \stp

\item  Use your expressions to shown that to first order in $1/c$
   \begin{align}
       \A_{\rm lrnz}(\r) =&  \int d^3\r_0 \frac{{\bm j}(\r_0)/c }{4\pi |\r - \r_0|} \\
       \A_{\rm coul}(\r) =&  \int d^3\r_0 \frac{ {\bm j}(\r_0)/c + {\bm j}_{\rm D}(\r_0)/c}{4\pi |\r-\r_0|}
   \end{align}
   Evaluate the Lorenz gauge integral (using the results of homework 2)
      yielding
      \st
            \label{Aguess}
            {\bm A}_{\rm lrnz}  = \frac{\dot \p }{4\pi r c}
      \stp


\item  Show that the Coulomb gauge expression  
   can be written
      \begin{align}
     \A_{\rm coul}(\r) =&  
             \frac{\dot {\bm p} }{4\pi r c} 
             - \nabla \, \frac{\partial}{c \partial t} 
             \left[ \int d^3\r_0 \, \varphi(\r_0) \, 
             \frac{1}{4\pi|\r - \r_0|} \right] 
      \end{align}
      At home (or in class if you have time) show that\footnote{ Use the ``Coulomb Identity" 
   \st
      \frac{1}{4\pi|\r - \r_0|}  = \sum_{\ell m} \frac{1}{2\ell + 1} \, \frac{r_{<}^\ell}{ r_>^{\ell+1} } \,  Y_{\ell m}(\theta,\phi)  Y_{\ell m}(\theta_0,\phi_0)
   \stp
      and the trick in \href{http://tonic.physics.sunysb.edu/~dteaney/S18_Phy505/lectures/magneto_sphere.pdf}{lecture} with 
      \st
       \varphi(\r_0) =  \frac{p \cos(\theta_0) }{4\pi r_0^2 }
      \stp}
      \st
      \left[ \int d^3\r_0 \, \varphi(\r_0) \, 
             \frac{1}{4\pi|\r - \r_0|} \right]  = \frac{\p \cdot \n}{8\pi}
      \stp
      where $\n \equiv \hat{\r}$,  to show that
      \begin{subequations}
         \label{Acoul}
      \begin{align}
     \A_{\rm coul}(\r) =&   
             \frac{ \dot{\bm p} }{4\pi r c} -  \nabla \frac{\partial}{c \partial t} \left( \frac{\n \cdot \p}{8\pi } \right) \\
%         =&
%             \frac{ \dot{\bm p} }{4\pi r c} +  \frac{\n (\n\cdot \dot \p) - \dot \p}{8\pi r c}  \\
         =& \frac{ \n (\n \cdot \dot \p) + \dot \p}{8\pi rc } 
     \end{align}
      \end{subequations}
      As a by product you should find 
      \st
        \partial_i n_j = \frac{\delta_{ij} - n_i n_j}{r}
      \stp
      which will be relatively useful going forward.
      

   \item Without calculation explain why the magnetic field $\B^{(1)}$ from \Eq{Acoul} and \Eq{Aguess} must agree.  Which gauge is easier for the magnetic field?

   \item Starting from the equations written down in (a), determine the correction to  order $1/c^2$ to $\varphi$ and in the Lorenz and Coulomb gauges. You should find :  
      \st
           \varphi^{(2)}_{\rm lrnz} =  - \, \frac{\n \cdot \ddot \p }{8\pi r c^2}
      \stp
      Relate the two results for $(\varphi, \A) $ via a gauge transformation. 

   \item Determine the electric field to second order in $1/c$ using
      the Lorenz and Coulomb gauges. Notice how particularly
      simple the  Coulomb gauge is for this purpose.  You
      should find (in either gauge) 
      \st
        \E^{(2)} =  - \frac{ \n (\n\cdot \ddot \p) + \ddot \p }{8\pi r c^2} 
      \stp

    \item At what radius does $\E^{(2)}$ become comparable to $\E^{(0)}$

\end{enumerate}




\end{document}



